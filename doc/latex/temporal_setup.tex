%=====================================================================
% FILE: doc/latex/temporal_setup.tex
% INTENT:
%   Detailed documentation for configuring steady-state versus time-
%   varying runs in the AMPS Geospace SEP/GCR interface.
%
% METHODS / DESIGN:
%   - Paper-style narrative with concrete guidance tied to the wizard UI.
%   - Defines the three temporal modes implemented/anticipated.
%   - Documents the driving-file format used for time-series runs.
%
% IMPLEMENTATION NOTES:
%   - This PDF is linked from the in-app Docs menu.
%   - Compiles with pdflatex; keep dependencies minimal.
%   - Output is copied to ../temporal_setup.pdf.
%
% LAST UPDATED: 2026-02-21
%=====================================================================
\documentclass[11pt]{article}
\usepackage[margin=1in]{geometry}
\usepackage{amsmath,amssymb}
\usepackage{hyperref}
\usepackage{siunitx}
\usepackage{enumitem}
\usepackage{verbatim}
\setlist{nosep}

\title{Configuring Steady-State vs. Time-Varying Runs in the AMPS Wizard}
\author{AMPS Interface Documentation}
\date{\today}

\begin{document}
\maketitle

\section{Why the ``Temporal Mode'' matters}
In AMPS geospace transport calculations, the particle equation of motion depends on the background fields
\begin{equation}
  \frac{d\mathbf{p}}{dt} = q\,\bigl(\mathbf{E}(\mathbf{r},t) + \mathbf{v}\times\mathbf{B}(\mathbf{r},t)\bigr),
\end{equation}
so any time variability in $\mathbf{E}$ and/or $\mathbf{B}$ must be represented consistently.
The wizard provides a \emph{Temporal Mode} (Step 6) that selects how $t$ enters the problem.

\section{Temporal modes supported by the interface}
\subsection{STEADY\_STATE (single epoch snapshot)}
Use this mode when you want a static field configuration.
\begin{itemize}
  \item Choose one epoch (date/time) and a field model in Step 3.
  \item AMPS uses \emph{one} set of geomagnetic driving parameters for the entire run.
  \item Best for: cutoff maps, parameter sweeps, and controlled comparisons.
\end{itemize}
\textbf{Wizard behavior:} the time-series form is hidden; the output preview prints
\texttt{TEMPORAL\_MODE = STEADY\_STATE}.

\subsection{TIME\_SERIES (driven by an external time history)}
Use this mode when you need storm-time dynamics (changing pressure/IMF/Dst, etc.).
\begin{itemize}
  \item The field model is updated at a cadence \texttt{FIELD\_UPDATE\_DT}.
  \item Particle injection (or evaluation) occurs at \texttt{INJECT\_DT}.
  \item \texttt{INJECT\_DT} must be an integer multiple of \texttt{FIELD\_UPDATE\_DT}.
\end{itemize}
\textbf{Wizard behavior:} Step 6 shows a timeline illustration of the relationship between the two cadences.
The interface warns when \texttt{INJECT\_DT < FIELD\_UPDATE\_DT}.

\subsection{MHD\_COUPLED (planned / placeholder)}
This mode represents fully time-dependent fields supplied by an MHD model (e.g., BATS-R-US or GAMERA).
It is included in the UI as a forward-looking option; the implementation depends on the field I/O pipeline
and is typically not available in early CCMC prototype deployments.

\section{Driving sources for TIME\_SERIES}
The interface provides multiple ways to specify time histories:
\begin{enumerate}[label=\arabic*.]
  \item \textbf{OMNIWeb auto-fetch (UI simulation):} the wizard demonstrates the intended pipeline
        (fetch solar wind, fetch indices, merge, produce a driving file). In production, a back-end service
        performs the actual fetch on submission.
  \item \textbf{Upload a driving file:} you supply a text file with one epoch per row.
  \item \textbf{Scalar input (debug):} use a constant set of scalars but still exercise the time-series code path.
\end{enumerate}

\section{Time-series driving file format (\texttt{ts05\_driving.txt})}
For TS05-style scalar-driven models, AMPS expects one line per epoch.
A common format is:
\begin{verbatim}
# YYYY MM DD HH MM  Dst[nT]  Pdyn[nPa]  Bz[nT]  Vx[km/s]  Nsw[cm^-3]  By[nT]  Bx[nT]
2017 09 07 00 00   -20.0     2.3      -3.5     -420     6.1        2.0     1.0
2017 09 07 00 05   -21.0     2.4      -3.7     -430     6.2        2.1     1.0
...
\end{verbatim}
\paragraph{Rules and recommendations}
\begin{itemize}
  \item Timestamps must be strictly increasing.
  \item Use UTC unless explicitly stated otherwise.
  \item If your cadence is coarser than \texttt{FIELD\_UPDATE\_DT}, AMPS must interpolate or hold values.
        Prefer to provide data at the intended update cadence.
  \item If gaps exist, document how they were filled (linear interpolation is common).
\end{itemize}

\section{Choosing cadences: practical guidance}
\begin{itemize}
  \item \textbf{Field update cadence (\texttt{FIELD\_UPDATE\_DT}):} choose based on how quickly the field model changes.
        For storm-time studies, \SI{1}{\minute} to \SI{5}{\minute} is typical.
  \item \textbf{Injection cadence (\texttt{INJECT\_DT}):} choose based on how often you want new particles injected
        (or how often you want diagnostics written). \SI{15}{\minute} to \SI{60}{\minute} is common.
  \item Make \texttt{INJECT\_DT} a multiple of \texttt{FIELD\_UPDATE\_DT} to avoid ambiguous partial steps.
\end{itemize}

\section{Worked configuration examples}
\subsection{Example A: steady-state cutoff map}
\begin{itemize}
  \item Temporal Mode: \textbf{STEADY\_STATE}
  \item Field model: T96 or T15 snapshot at a single epoch
  \item Output domain: spherical shells at multiple altitudes
\end{itemize}

\subsection{Example B: Sep 2017 storm time-series}
\begin{itemize}
  \item Temporal Mode: \textbf{TIME\_SERIES}
  \item \texttt{FIELD\_UPDATE\_DT = 5 min}, \texttt{INJECT\_DT = 30 min}
  \item Driving: \texttt{ts05\_driving.txt} generated from OMNI + Dst
  \item Trajectory evaluation: along a spacecraft ephemeris (see the Output Domain doc)
\end{itemize}

\section{References (starting points)}
\begin{itemize}
  \item Tsyganenko, N. A., and Sitnov, M. I. (2005): TS05 storm-time geomagnetic field model.
  \item Smart, D. F., and Shea, M. A. (2009): Geomagnetic cutoff rigidities and trajectory tracing.
\end{itemize}

\end{document}
