% ==============================================================================
% FILE:    cutoff_rigidity.tex
% PROJECT: AMPS CCMC SEP/GCR Transport Interface
% PURPOSE: "Paper-style" documentation for geomagnetic cutoff rigidity.
%
% BUILD:
%   pdflatex -interaction=nonstopmode -halt-on-error cutoff_rigidity.tex
%
% OUTPUT (copied to ../):
%   ../cutoff_rigidity.pdf
%
% NOTES:
%   - This document is written to be readable as a standalone reference.
%   - Equations are intentionally kept compact and focused on what is needed
%     to understand (and validate) the code path used by the web interface.
% ============================================================================== 
\documentclass[11pt]{article}
\usepackage[margin=1in]{geometry}
\usepackage{amsmath,amssymb}
\usepackage{hyperref}
\usepackage{siunitx}
\usepackage{enumitem}
\setlist{nosep}

\title{Geomagnetic Cutoff Rigidity in AMPS: Definitions, Penumbra, and Backtracing}
\author{AMPS Interface Documentation}
\date{\today}

\begin{document}
\maketitle

\section{What is cutoff rigidity?}
Charged particles moving in the geomagnetic field are filtered by their \emph{magnetic rigidity}
\begin{equation}
  R \equiv \frac{p c}{Z e},
\end{equation}
where $p$ is particle momentum, $Z$ is charge state, $e$ is the elementary charge, and $c$ is the speed of light.
For a given location and arrival direction, there exists a threshold rigidity below which trajectories are typically \emph{forbidden} (do not connect to interplanetary space) and above which they are \emph{allowed}.

\paragraph{Why rigidity instead of energy?}
Magnetic deflection depends on momentum per unit charge.
Energy-based thresholds depend on particle species through the $p(E)$ relation and $Z$.

\section{Trajectory-based definition}
AMPS evaluates cutoffs using \emph{numerical trajectory tracing}:
\begin{itemize}
  \item Choose an observation point $\mathbf{r}_0$ and an arrival direction $\hat{\mathbf{v}}_0$ (in GSM unless noted).
  \item For a trial rigidity $R$, construct the corresponding momentum and velocity for the chosen particle species.
  \item Integrate the Lorentz equation backward in time (equivalently forward for antiparticles):
  \begin{equation}
    \frac{d\mathbf{p}}{dt} = q\,\bigl(\mathbf{E}(\mathbf{r},t) + \mathbf{v}\times\mathbf{B}(\mathbf{r},t)\bigr).
  \end{equation}
  \item Apply stop conditions (magnetopause crossing, maximum time, or return to atmosphere/inner boundary).
\end{itemize}
A trajectory is labeled \emph{allowed} if it reaches the upstream boundary (solar wind / magnetopause) without intersecting the inner loss boundary.

\section{Penumbra and effective cutoff}
Realistic geomagnetic fields produce a \emph{penumbra}: alternating allowed/forbidden bands in rigidity.
Following classic practice (e.g., Cooke et\,al., 1991; Smart \& Shea, 2009), define:
\begin{itemize}
  \item $R_L$: the \emph{lower} cutoff (below which all trajectories are forbidden),
  \item $R_U$: the \emph{upper} cutoff (above which all trajectories are allowed).
\end{itemize}
Between $R_L$ and $R_U$ the transmissivity is fractional.
The \emph{effective cutoff} is computed by summing allowed bands in the penumbra:
\begin{equation}
  R_C \;=\; R_L + \sum_{k=1}^{N_{\rm allowed}} \Delta R_k,
\end{equation}
where $\Delta R_k$ are the rigidity widths of allowed intervals inside $(R_L, R_U)$.
In discrete scanning, this becomes a weighted count of allowed samples.

\section{Vertical vs. directional cutoffs}
A \emph{vertical cutoff} uses $\hat{\mathbf{v}}_0$ aligned with the local zenith.
For anisotropic SEP events and LEO particle telescopes, AMPS supports \emph{directional cutoffs} using the instrument look direction.
This matters because $R_C$ can vary strongly with pitch/azimuth in the penumbra.

\section{Implementation notes (AMPS interface conventions)}
\begin{itemize}
  \item Coordinates: the interface uses GSM for fields and particle state unless explicitly specified.
  \item Boundary surfaces: the magnetopause (e.g., Shue) is used as an ``escape'' boundary; crossing it ends the tracing.
  \item Time dependence: temporal modes (static, driven) are handled by updating $\mathbf{E}(t)$ and $\mathbf{B}(t)$ during integration.
  \item Output: AMPS reports $R_L$, $R_U$, and $R_C$ when penumbra scanning is enabled.
\end{itemize}

\section{References (selected)}
\begin{itemize}
  \item Cooke, D. J., et al. (1991). ``On cosmic-ray cut-off terminology.'' \emph{Nuovo Cimento C}.
  \item Smart, D. F., \& Shea, M. A. (2009). ``Fifty years of progress in geomagnetic cutoff rigidity determinations.'' \emph{Adv. Space Res.}
  \item Kress, B. T., et al. (2015). ``Modeling geomagnetic cutoffs for space weather applications.'' \emph{J. Geophys. Res. Space Physics}.
\end{itemize}

\end{document}
