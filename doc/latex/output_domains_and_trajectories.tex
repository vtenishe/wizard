%=====================================================================
% FILE: doc/latex/output_domains_and_trajectories.tex
% INTENT:
%   Detailed documentation for Step 7 (Output Domain): how to define
%   individual points, trajectories, and spherical shells, including
%   example trajectory-file formats.
%
% METHODS / DESIGN:
%   - Maps directly onto the wizard UI cards: POINTS / TRAJECTORY / SHELLS.
%   - Specifies coordinate frames, units, and practical validation tips.
%   - Provides multiple simple text formats for trajectory upload.
%
% IMPLEMENTATION NOTES:
%   - This PDF is linked from Docs and should open directly as a .pdf.
%   - Compiles with pdflatex.
%   - Output is copied to ../output_domains_and_trajectories.pdf.
%
% LAST UPDATED: 2026-02-21
%=====================================================================
\documentclass[11pt]{article}
\usepackage[margin=1in]{geometry}
\usepackage{amsmath,amssymb}
\usepackage{hyperref}
\usepackage{siunitx}
\usepackage{enumitem}
\usepackage{verbatim}
\setlist{nosep}

\title{Output Domains in the AMPS Wizard: Points, Trajectories, and Spherical Shells}
\author{AMPS Interface Documentation}
\date{\today}

\begin{document}
\maketitle

\section{Concept: what is an ``Output Domain''?}
The Output Domain (Step 7) defines \emph{where} AMPS evaluates geomagnetic cutoffs, particle access, and flux.
Conceptually, AMPS evaluates a map
\begin{equation}
  \bigl(\mathbf{r}, \hat{\mathbf{v}}, t, E\bigr) \;\mapsto\; \text{allowed/forbidden},\; R_c,\; J(E),\; \Phi(E> E_0),\;\ldots
\end{equation}
for a set of sampling locations $\mathbf{r}$ (and optionally directions $\hat{\mathbf{v}}$) at one or more times $t$.

\paragraph{Coordinate system}
Unless stated otherwise, the interface assumes \textbf{GSM} coordinates.
If you provide data in another frame (GSE, GEO, SM), convert before upload.

\paragraph{Units}
The wizard UI labels distances in \textbf{$R_E$} (Earth radii) for geospace problems.
If your data are in kilometers, convert using $1\,R_E \approx \SI{6371}{km}$.

\section{Mode 1: Individual Points (POINTS)}
Use POINTS when you have a small set of discrete locations (stations, grid points, spacecraft points-of-interest).
\begin{itemize}
  \item Provide $N$ points, each as $(x,y,z)$ in GSM.
  \item Optional metadata (label, altitude) may be included but is ignored by the simplest parser.
  \item Best for: validation, site-specific cutoffs, ``spot checks''.
\end{itemize}

\subsection{Example point list (space-delimited)}
\begin{verbatim}
# x(Re)   y(Re)   z(Re)    label
 6.60     0.00    0.00     GEO
 1.05     0.00    0.00     LEO_equator
 1.00     0.00    0.50     LEO_midlat
\end{verbatim}

\section{Mode 2: Trajectories (TRAJECTORY)}
Use TRAJECTORY when you want sampling along an orbit or path (e.g., Van Allen Probes, ISS, a flight segment).
A trajectory is a time-ordered list of positions.
\begin{itemize}
  \item Each row provides time + position.
  \item Time is used when Temporal Mode is TIME\_SERIES; otherwise it may be ignored.
  \item Positions should be monotonic in time (no backward jumps).
\end{itemize}

\subsection{Trajectory file formats: recommended options}
To keep the interface robust, support a small set of easy-to-parse ASCII formats.
All formats below are \textbf{plain text}, one sample per line.

\subsubsection{Format A (Gregorian columns) --- recommended}
\begin{verbatim}
# YYYY MM DD HH MM SS   x(Re)   y(Re)   z(Re)
2017 09 07 00 00 00     4.12    -0.52    0.80
2017 09 07 00 01 00     4.10    -0.50    0.81
...
\end{verbatim}
\textbf{Notes:} seconds may be omitted if your cadence is 1 minute; if omitted, assume SS=0.

\subsubsection{Format B (ISO-8601 timestamp)}
\begin{verbatim}
# ISO_TIME_UTC               x(Re)   y(Re)   z(Re)
2017-09-07T00:00:00Z         4.12    -0.52    0.80
2017-09-07T00:01:00Z         4.10    -0.50    0.81
...
\end{verbatim}
\textbf{Notes:} include a trailing \texttt{Z} to indicate UTC.

\subsubsection{Format C (positions only; steady-state only)}
\begin{verbatim}
# x(Re)   y(Re)   z(Re)
4.12    -0.52    0.80
4.10    -0.50    0.81
...
\end{verbatim}
\textbf{Use only} when Temporal Mode is STEADY\_STATE and you do not need time tagging.

\subsection{Validation tips}
\begin{itemize}
  \item Plot $r=\sqrt{x^2+y^2+z^2}$ versus time to catch unit mistakes (km vs $R_E$).
  \item Verify GSM orientation: in GSM, +X points sunward.
  \item If you see discontinuities, check that your time column is strictly increasing.
\end{itemize}

\section{Mode 3: Spherical Shells (SHELLS)}
Use SHELLS when you want a global (or regional) map at fixed altitude(s).
\begin{itemize}
  \item Define one or more radii $r = R_E + h$ (or directly $r$ in $R_E$).
  \item Choose angular resolution (e.g., $\Delta\lambda$, $\Delta\phi$ or number of points).
  \item Best for: global cutoff maps, exposure maps, ``LEO shell'' diagnostics.
\end{itemize}

\subsection{Example shell specification}
\begin{verbatim}
# shell radii (Re)
1.02
1.05
1.10
# grid: 1-degree lat/lon
LAT_STEP_DEG 1
LON_STEP_DEG 1
\end{verbatim}

\section{How sampling is used in the calculations}
For each sample (point/trajectory location/shell node), AMPS computes a cutoff rigidity (or access) by backtracing
trajectories in the configured fields. Once $R_c$ is known, the chosen upstream spectrum $J(E)$ is filtered to produce
local flux and integral quantities.

\section{References (starting points)}
\begin{itemize}
  \item Smart, D. F., and Shea, M. A. (2009): trajectory tracing and cutoffs.
  \item Example mission: Van Allen Probes ephemeris products (for realistic trajectories).
\end{itemize}

\end{document}
