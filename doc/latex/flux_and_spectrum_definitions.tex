%=====================================================================
% FILE: doc/latex/flux_and_spectrum_definitions.tex
% INTENT:
%   LaTeX source for a documentation PDF linked from the in-app Docs
%   menu. The PDF is intended to be read alongside the wizard and
%   provide paper-style detail: definitions, equations, references,
%   and worked examples.
%
% METHODS / DESIGN:
%   - Keep structure consistent across docs: Abstract → Background →
%     Methods → Implementation Notes → Example(s) → References.
%   - Use clear variable definitions and units.
%   - Prefer short, self-contained equations over long derivations.
%
% IMPLEMENTATION NOTES:
%   - Generated PDFs live in doc/*.pdf; sources are in doc/latex/*.tex.
%   - Keep equations compatible with pdflatex.
%
% LAST UPDATED: 2026-02-21
%=====================================================================
% ==============================================================================
% FILE:    flux_and_spectrum_definitions.tex
% PROJECT: AMPS CCMC SEP/GCR Transport Interface
% PURPOSE: Definitions of flux / intensity / spectra and unit conventions.
%
% BUILD:
%   pdflatex -interaction=nonstopmode -halt-on-error flux_and_spectrum_definitions.tex
% OUTPUT:
%   ../flux_and_spectrum_definitions.pdf
% ============================================================================== 
\documentclass[11pt]{article}
\usepackage[margin=1in]{geometry}
\usepackage{amsmath,amssymb}
\usepackage{hyperref}
\usepackage{siunitx}
\usepackage{enumitem}
\setlist{nosep}

\title{Flux and Spectrum Definitions Used by AMPS}
\author{AMPS Interface Documentation}
\date{\today}

\begin{document}
\maketitle

\section{Differential intensity (``differential flux'')}
Throughout the AMPS interface, the particle spectrum is specified as a \emph{differential directional intensity}
\begin{equation}
  J(E) \equiv \frac{dN}{dA\,dt\,d\Omega\,dE},
\end{equation}
with units typically
\begin{equation}
  J(E)\;[\text{p}~\text{cm}^{-2}\,\text{s}^{-1}\,\text{sr}^{-1}\,(\text{MeV}/\text{n})^{-1}].
\end{equation}
If you are working in SI, replace cm$^{-2}$ with m$^{-2}$.

\paragraph{Energy per nucleon.}
For ions, AMPS commonly uses $E$ in \si{MeV/nucleon}. This makes spectral shapes comparable across species.

\section{Integral flux and pfu}
NOAA/SWPC frequently reports integral proton fluxes above a threshold energy,
\begin{equation}
  F(>E_0) = \int_{E_0}^{\infty} J(E)\,dE,
\end{equation}
often in ``pfu'' where
\begin{equation}
  1~\text{pfu} = 1~\text{proton}~\text{cm}^{-2}\,\text{s}^{-1}\,\text{sr}^{-1} \quad \text{(integral)}.
\end{equation}
When fitting spectra to integral channels, AMPS performs the above integral of the chosen model.

\section{Rigidity spectra and variable changes}
Some literature expresses spectra in rigidity $R$:
\begin{equation}
  J_R(R) = \frac{dN}{dA\,dt\,d\Omega\,dR}.
\end{equation}
Conversion uses $J_R(R) = J_E(E)\,\left|\frac{dE}{dR}\right|$.
Because $R = pc/(Ze)$, $E(R)$ depends on species mass and charge.

\section{Omnidirectional flux}
The omnidirectional differential flux is
\begin{equation}
  j(E) = \int J(E,\Omega)\,d\Omega.
\end{equation}
If the distribution is isotropic, $j(E)=4\pi J(E)$.
AMPS supports anisotropic source models, so the $4\pi$ factor should only be used when justified.

\section{How the interface expects spectra}
\begin{itemize}
  \item You select a spectrum \emph{model} (power law, cutoff, Band, LIS force-field, or table).
  \item You enter parameters in the units displayed by the UI.
  \item The preview shows $J(E)$, and AMPS expects $J(E)$ in the input file.
\end{itemize}

\section{References (selected)}
\begin{itemize}
  \item Grieder, P. K. F. (2001). \emph{Cosmic Rays at Earth}. (Rigidity and intensity conventions.)
  \item NOAA/SWPC. ``GOES Proton Flux'' product documentation (pfu conventions).
\end{itemize}

\end{document}
