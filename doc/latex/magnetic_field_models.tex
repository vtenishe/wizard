%=====================================================================
% FILE: doc/latex/magnetic_field_models.tex
% INTENT:
%   LaTeX source for a documentation PDF linked from the in-app Docs
%   menu. The PDF is intended to be read alongside the wizard and
%   provide paper-style detail: definitions, equations, references,
%   and worked examples.
%
% METHODS / DESIGN:
%   - Keep structure consistent across docs: Abstract → Background →
%     Methods → Implementation Notes → Example(s) → References.
%   - Use clear variable definitions and units.
%   - Prefer short, self-contained equations over long derivations.
%
% IMPLEMENTATION NOTES:
%   - Generated PDFs live in doc/*.pdf; sources are in doc/latex/*.tex.
%   - Keep equations compatible with pdflatex.
%
% LAST UPDATED: 2026-02-21
%=====================================================================
% ==============================================================================
% FILE:    magnetic_field_models.tex
% PROJECT: AMPS CCMC SEP/GCR Transport Interface
% PURPOSE: Documentation of magnetic field models exposed in the interface.
%
% BUILD:
%   pdflatex -interaction=nonstopmode -halt-on-error magnetic_field_models.tex
% OUTPUT:
%   ../magnetic_field_models.pdf
% ============================================================================== 
\documentclass[11pt]{article}
\usepackage[margin=1in]{geometry}
\usepackage{amsmath,amssymb}
\usepackage{hyperref}
\usepackage{siunitx}
\usepackage{enumitem}
\setlist{nosep}

\title{Magnetic Field Models for Geospace Particle Tracing}
\author{AMPS Interface Documentation}
\date{\today}

\begin{document}
\maketitle

\section{Total field representation}
AMPS typically represents the magnetic field as
\begin{equation}
  \mathbf{B}(\mathbf{r},t) = \mathbf{B}_{\mathrm{int}}(\mathbf{r},t) + \mathbf{B}_{\mathrm{ext}}(\mathbf{r},t),
\end{equation}
where $\mathbf{B}_{\mathrm{int}}$ is the internal (main) field and $\mathbf{B}_{\mathrm{ext}}$ is the magnetospheric current system contribution.

\section{Internal field (IGRF / dipole)}
For most applications, $\mathbf{B}_{\mathrm{int}}$ is modeled using IGRF (or a dipole approximation for sensitivity testing).
The internal field provides the dominant contribution near Earth and controls low-altitude cutoffs.

\section{Empirical external field models (Tsyganenko family)}
Empirical models represent $\mathbf{B}_{\mathrm{ext}}$ using parametrizations constrained by spacecraft observations.
Typical inputs are solar-wind dynamic pressure, IMF $B_y/B_z$, Dst, and/or Kp.

\subsection*{T96}
Tsyganenko (1996) provides a widely used external field model suitable for many cutoff studies.

\subsection*{TS05 / T15}
Storm-time models incorporate additional drivers and are designed to capture disturbed intervals.
These are often preferred for event-realistic SEP cutoff calculations.

\section{MHD-driven fields (BATSRUS / GAMERA)}
For highest fidelity, AMPS can ingest a 3D time-dependent magnetic field from a global MHD model.
In this mode:
\begin{itemize}
  \item The MHD solution provides $\mathbf{B}(\mathbf{r},t)$ (and optionally $\mathbf{E}$).
  \item Particle tracing uses spatio-temporal interpolation on the MHD grid.
  \item Boundary surfaces (magnetopause, inner boundary) should be consistent with the MHD domain.
\end{itemize}
This option is computationally heavier but can reproduce event-specific structures not captured by analytic fits.

\section{Notes for cutoff calculations}
\begin{itemize}
  \item Using a storm-time external model (e.g., TS05/T15) can change $R_C$ by several GV at mid-latitudes.
  \item Time dependence can be critical during rapid Dst changes; ensure $\Delta t$ in tracing resolves the driver cadence.
  \item Consistency with the electric field model matters for drift trajectories, especially in the inner magnetosphere.
\end{itemize}

\section{References (selected)}
\begin{itemize}
  \item Tsyganenko, N. A. (1996). ``Effects of the solar wind conditions on the global magnetosphere configuration...'' \emph{JGR}.
  \item Tsyganenko, N. A., \& Sitnov, M. I. (2005). (TS05). \emph{JGR}.
  \item Tsyganenko, N. A., \& Andreeva, V. A. (2015). (T15). \emph{JGR}.
\end{itemize}

\end{document}
