% ==============================================================================
% FILE:    electric_field_models.tex
% PROJECT: AMPS CCMC SEP/GCR Transport Interface
% PURPOSE: Documentation of electric field models exposed in the interface.
%
% BUILD:
%   pdflatex -interaction=nonstopmode -halt-on-error electric_field_models.tex
% OUTPUT:
%   ../electric_field_models.pdf
% ============================================================================== 
\documentclass[11pt]{article}
\usepackage[margin=1in]{geometry}
\usepackage{amsmath,amssymb}
\usepackage{hyperref}
\usepackage{siunitx}
\usepackage{enumitem}
\setlist{nosep}

\title{Electric Field Models Used for Particle Tracing}
\author{AMPS Interface Documentation}
\date{\today}

\begin{document}
\maketitle

\section{Decomposition}
In AMPS geospace transport runs, the total electric field is represented as a superposition
\begin{equation}
  \mathbf{E}(\mathbf{r},t) = \mathbf{E}_{\mathrm{cor}}(\mathbf{r},t) + \mathbf{E}_{\mathrm{conv}}(\mathbf{r},t) + \mathbf{E}_{\mathrm{ind}}(\mathbf{r},t),
\end{equation}
where the terms correspond to corotation, large-scale convection, and inductive (time-varying $\mathbf{B}$) contributions.
The interface allows you to enable/disable each contribution depending on the model combination you need.

\section{Corotation electric field}
Corotation is the electric field in the inertial frame associated with plasma corotating with Earth:
\begin{equation}
  \mathbf{E}_{\mathrm{cor}} = - (\boldsymbol{\omega} \times \mathbf{r}) \times \mathbf{B},
\end{equation}
with Earth's rotation rate $\omega=\SI{7.2921159e-5}{rad\,s^{-1}}$.
This term is most relevant in the inner magnetosphere and is frequently included when tracing particles that drift near Earth.

\section{Volland--Stern convection (Kp-parameterized)}
A commonly used robust model for global convection is the Volland--Stern potential.
A simple form for the electric potential in the equatorial plane is
\begin{equation}
  \Phi(r,\theta) = A\,r^{\gamma}\,\sin\theta,
\end{equation}
where $r$ is radial distance, $\theta$ is the azimuthal angle, $\gamma$ controls shielding, and $A$ controls intensity.
The convection electric field is then
\begin{equation}
  \mathbf{E}_{\mathrm{conv}} = -\nabla\Phi.
\end{equation}
In the interface, $A$ is derived from Kp using standard empirical relations; $\gamma$ is user-controlled.

\section{Weimer (solar-wind / IMF driven)}
Weimer-type models are data-driven empirical reconstructions of the ionospheric electric potential as a function of IMF and solar-wind conditions.
In practice, AMPS uses the mapped electric field/potential as a time-dependent driver for more event-realistic runs.
Typical inputs include IMF $B_y$, $B_z$, solar-wind speed, and dipole tilt.

\section{Inductive electric field from $d\mathbf{B}/dt$}
When the magnetic field varies in time, Maxwell--Faraday requires
\begin{equation}
  \nabla\times \mathbf{E}_{\mathrm{ind}} = -\frac{\partial \mathbf{B}}{\partial t}.
\end{equation}
A physically consistent minimal model (used when $\partial\mathbf{B}/\partial t$ is spatially uniform in the region of interest) is
\begin{equation}
  \mathbf{E}_{\mathrm{ind}}(\mathbf{r}) = -\frac{1}{2}\,\mathbf{r}\times\frac{d\mathbf{B}}{dt}.
\end{equation}
This choice satisfies $\nabla\times\mathbf{E} = -d\mathbf{B}/dt$ and is unique up to an added gradient field (gauge).

\paragraph{Inputs in the interface.}
The UI accepts $dB_x/dt$, $dB_y/dt$, $dB_z/dt$ in GSM with units of \si{nT/min}. Internally you should convert to \si{T/s}.

\section{References (selected)}
\begin{itemize}
  \item Volland, H. (1973). ``A semiempirical model of large-scale magnetospheric electric fields.'' \emph{JGR}.
  \item Stern, D. P. (1975). ``The motion of a proton in the equatorial magnetosphere.'' \emph{JGR}.
  \item Weimer, D. R. (2005). ``Improved ionospheric electrodynamic models...'' \emph{JGR}.
\end{itemize}

\end{document}
