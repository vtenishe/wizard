% ==============================================================================
% FILE:    overview_calculation_pipeline.tex
% PROJECT: AMPS CCMC SEP/GCR Transport Interface
% PURPOSE: High-level description of how the end-to-end calculations are done:
%          - Upstream spectrum specification
%          - Trajectory tracing through E/B fields
%          - Rigidity cutoff computation
%          - Flux mapping / attenuation to LEO
%          - Optional reconstruction of upstream spectra from GOES
%
% BUILD:
%   pdflatex -interaction=nonstopmode -halt-on-error overview_calculation_pipeline.tex
% OUTPUT:
%   ../overview_calculation_pipeline.pdf
% ============================================================================== 
\documentclass[11pt]{article}
\usepackage[margin=1in]{geometry}
\usepackage{amsmath,amssymb}
\usepackage{hyperref}
\usepackage{siunitx}
\usepackage{enumitem}
\setlist{nosep}

\title{AMPS Geospace Radiation Pipeline: From Magnetopause Spectrum to LEO Flux and Cutoffs}
\author{AMPS Interface Documentation}
\date{\today}

\begin{document}
\maketitle

\section{Pipeline overview}
A typical run uses the following stages:
\begin{enumerate}
  \item Specify particle species and an upstream spectrum $J(E,t)$ at the outer boundary (usually the magnetopause).
  \item Choose magnetic and electric field models $\mathbf{B}(\mathbf{r},t)$ and $\mathbf{E}(\mathbf{r},t)$.
  \item Define the domain boundary (e.g., Shue magnetopause or a box) and inner loss boundary.
  \item Perform trajectory tracing (forward for transport, or backward for cutoffs) through the fields.
  \item Derive observables: cutoff rigidity $R_C$, access probability, and flux at the target altitude/location.
\end{enumerate}

\section{Equations of motion}
Particles are advanced with the relativistic Lorentz force:
\begin{align}
  \frac{d\mathbf{r}}{dt} &= \mathbf{v},\\
  \frac{d\mathbf{p}}{dt} &= q\,\bigl(\mathbf{E}(\mathbf{r},t) + \mathbf{v}\times\mathbf{B}(\mathbf{r},t)\bigr),\\
  \mathbf{p} &= \gamma m \mathbf{v}, \qquad \gamma = \left(1-\frac{v^2}{c^2}\right)^{-1/2}.
\end{align}

\section{Cutoff rigidity computation (backtracing)}
To compute cutoffs at a target location, AMPS backtraces particles launched with selected rigidities $R$ and direction $\hat{\mathbf{v}}$.
For each $R$, the outcome is classified as \emph{allowed} or \emph{forbidden} based on whether the trajectory reaches the outer boundary without intersecting the loss boundary.
Scanning across $R$ yields $R_L$, $R_U$, and the effective cutoff $R_C$.

\section{Flux mapping}
Given an upstream spectrum $J(E)$, the flux at a target can be estimated using transmissivity / access probability $T(E,\Omega)$:
\begin{equation}
  J_{\mathrm{target}}(E) = \int T(E,\Omega)\,J_{\mathrm{up}}(E,\Omega)\,d\Omega.
\end{equation}
In a deterministic backtracing approach, $T$ is approximated by the fraction of launched trajectories that are allowed for a given $(E,\Omega)$.

\section{Reconstructing upstream spectra from GOES}
If you want the upstream boundary condition to be event-specific, you can fit a parametric model $J(E;\boldsymbol{\theta}(t))$ to GOES channel data.
The forward model integrates $J(E)$ over each channel's energy range (or response function), and the fit is repeated over time.

\section{Outputs}
Typical outputs include:
\begin{itemize}
  \item $R_L$, $R_U$, and $R_C$ (cutoff rigidity),
  \item flux maps and time series at selected altitudes/locations,
  \item access probability vs. energy/rigidity,
  \item diagnostic trajectories.
\end{itemize}

\end{document}
